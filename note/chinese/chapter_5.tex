%%%%%%%%%%%%%%%%%%%%%%%%%%%%%%%%%%%%%%%%%%%%%%%%%%%%%%%%%%%%%%%%%%%%%%%%%%%%%%%%%%%%%%%%%%%%%%%%%%%%%%%%%%%%%%%%%%%%%%%%%%%%%%%%%%%%%%%%%%%%%%%%%%%%%%%%%%%%%%%%%%%%%%%%

%%%%%%%%%%%%%%%%%%%%%%%%%%%%%%%%%%%%%%%%%%%%%%%%%%%%%%%%%%%%%%%%%%%%%%%%   Author:Yao Zhang  %%%%%%%%%%%%%%%%%%%%%%%%%%%%%%%%%%%%%%%%%%%%%%%%%%%%%%%%%%%%%%%%%%%%%%%%%%%
%%%%%%%%%%%%%%%%%%%%%%%%%%%%%%%%%%%%%%%%%%%%%%%%%%%%%%%%%%%%%%%%%%%%%% Email: jaafar_zhang@163.com  %%%%%%%%%%%%%%%%%%%%%%%%%%%%%%%%%%%%%%%%%%%%%%%%%%%%%%%%%%%%%%%%%%%%

%%%%%%%%%%%%%%%%%%%%%%%%%%%%%%%%%%%%%%%%%%%%%%%%%%%%%%%%%%%%%%%%%%%%%%%%%%%%%%%%%%%%%%%%%%%%%%%%%%%%%%%%%%%%%%%%%%%%%%%%%%%%%%%%%%%%%%%%%%%%%%%%%%%%%%%%%%%%%%%%%%%%%%%%

\documentclass[11pt]{article}
\usepackage[utf8]{inputenc} 
\usepackage[table]{xcolor}
\usepackage[most]{tcolorbox}
\usepackage[left=2.50cm, right=1.50cm, top=2.0cm, bottom=2.50cm]{geometry}
\usepackage{xcolor,url}
\usepackage{amsmath,amsthm,amsfonts,amssymb,amscd,multirow,booktabs,fullpage,calc,multicol}
\usepackage{lastpage,enumitem,fancyhdr,mathrsfs,wrapfig,setspace,cancel,amsmath,empheq,framed}
\usepackage[retainorgcmds]{IEEEtrantools}
\newlength{\tabcont}
\setlength{\parindent}{0.0in}
\setlength{\parskip}{0.05in}
\colorlet{shadecolor}{orange!15}
\parindent 0in
\parskip 12pt
\geometry{margin=1in, headsep=0.25in}
\usepackage{subfig,graphicx,framed}
\graphicspath{ {img/ch5/} }
\usepackage{ctex}
\usepackage{txfonts}

%%%%%%%%%%%%%%%%%%%%%%%%%%%%%%%%%%%%%%%%%%%%%%%%%%%%%%%%%%%%%%%%%%%%%%%%%%%%%%%%%%%%%%%%%%%%%%%%%%%%%%%%%%%%%%%%%%%%%%%%%%%%%%%%%%%%%%%%%%%%%%%%%%%%%%%%%%%%%%%%%%%%%%%%

\newtheorem{theorem}{Theorem}[subsection]
\newtheorem{definition}{Definition}[subsection]
\newtheorem{exercise}{Exercise}[subsection]
\newtheorem{note}{Note}[subsection]
\newtheorem{notation}{Notation}
\newtheorem{lemma}{Lemma}[subsection]
\newtheorem{proposition}{Proposition}[subsection]
\newtheorem{example}{Example}[subsection]
\newtheorem{homework}{Homework}[section]
\newtheorem{summary}{Summary}[subsection]
\newtheorem{corollary}{Corollary}[subsection]
\newtheorem{remark}{Remark}[subsection]
\renewcommand{\cite}[1]{[#1]}
\makeatletter
\@addtoreset{equation}{subsection}
\makeatother
\renewcommand{\theequation}{\arabic{section}.\arabic{subsection}.\arabic{equation}}
\usepackage[colorlinks,linkcolor=blue, anchorcolor=green,citecolor=red]{hyperref}

\renewcommand{\contentsname}{\centering \textbf{\kaishu }}

\renewcommand{\figurename}{\textbf{\kaishu 图}}
%\begin{figure}[!htb]
%	\centering
%	\subfloat[$A \cap B$]{%
%		\includegraphics[width=0.3\linewidth,height=0.2\linewidth]{img001.jpg}}
%	\label{img001}\qquad \qquad %\hfill
%	\subfloat[${A_1} \cap {A_2} \cap {A_3}$]{%
%		\includegraphics[width=0.3\linewidth,height=0.2\linewidth]{img002.jpg}}
%	\label{img002}
	%\caption{ Examples.}
%\end{figure}
%\begin{figure}[!htb]
%	\centering
%	\includegraphics[width=0.4\linewidth,height=0.3\linewidth]{img005.jpg}
%	\label{img005}
	%\caption{ illustration for $ 3 $}
%\end{figure}
%%%%%%%%%%%%%%%%%%%%%%%%%%%%%%%%%%%%%%%%%%%%%%%%%%%%%%%%%%%%%%%%%%%%%%%%%%%%%%%%%%%%%%%%%%%%%%%%%%%%%%%%%%%%%%%%%%%%%%%%%%%%%%%%%%%%%%%%%%%%%%%%%%%%%%%%%%%%%%%%%%%%%%%%%%

\def\beginrefs{\begin{list}%
		{[\arabic{equation}]}{\usecounter{equation}
			\setlength{\leftmargin}{0.8truecm}\setlength{\labelsep}{0.4truecm}%
			\setlength{\labelwidth}{1.6truecm}}}
	\def\endrefs{\end{list}}
\def\bibentry#1{\item[\hbox{[#1]}]}

%%%%%%%%%%%%%%%%%%%%%%%%%%%%%%%%%%%%%%%%%%%%%%%%%%%%%%%%%%%%%%%%%%%%%%%%%%%%%%%%%%%%%%%%%%%%%%%%%%%%%%%%%%%%%%%%%%%%%%%%%%%%%%%%%%%%%%%%%%%%%%%%%%%%%%%%%%%%%%%%%%%%%%%%%%
\begin{document}
	\kaishu 
	
	\thispagestyle{empty}
	
	\setcounter{section}{4}
	
	\begin{center}
		{\LARGE \bf   \kaishu 向量分析 }
		
		\vspace{-0.25cm}
		
		{ \large \kaishu 林琦焜 }
		
		\vspace{-0.35cm}
		
		{ \small Re-edited by Yao Zhang}
	\end{center}
%%%%%%%%%%%%%%%%%%%%%%%%%%%%%%%%%%%%%%%%%%%%%%%%%%%%%%%%%%%%%%%%%%%%%%%%%%%%%%%%%%%%%%%%%%%%%%%%%%%%%%%%%%%%%%%%%%%%%%%%%%%%%%%%%%%%%%%%%%%%%%%%%%%%%%%%%%%%%%%%%%%%%%%%%%

%\vspace{-2.5cm}


%\tableofcontents

%\newpage 


%{\pagestyle{empty}\mbox{}\newpage\pagestyle{empty}}

%\newpage



%%%%%%%%%%%%%%%%%%%%%%%%%%%%%%%%%%%%%%%%%%%%%%%%%%%%%%%%%%%%%%%%%%%%%%%%%%%%%%%%%%%%%%%%%%%%%%%%%%%%%%%%%%%%%%%%%%%%%%%%%%%%%%%%%%%%%%%%%%%%%%%%%%%%%%%%%%%%%%%%%%%%%%%%%%

\section{\kaishu 向量分析之应用}

{\color{cyan} 数学家们可以怀疑他们得到的结果是否适用于立体几何学,流体静力学或电磁学. 但是, 正如我们现在所研究的自然哲学一样, 必然赋予我们的运算以某种形式, 对其每一步骤必须能够作出物理学的解释. 在这里, 我们的计算能力比较纯粹的计算原则的运用和对结果的解释更有用处. 
	
--- James Clerk Maxwell, 1831-1879}

正如微积分是为了解决牛顿力学自然而然发展出来的一门学问, 向量分析也是因为想要了解电磁学而应运而生的一门学问, 任何一门有活力的数学必定是为了解决问题而来.

{\color{cyan} 任何科学只要能产生充裕的问题, 就有丰富的生命力, 若问题匮乏, 就显示了死亡或发展停顿. 
	
 --- David Hilbert, 1862-1943}

我们在这一章简单地介绍流体力学与电磁学, 对于更深的材料则留给有兴趣的读者去查阅相关的题材. 


%%%%%%%%%%%%%%%%%%%%%%%%%%%%%%%%%%%%%%%%%%%%%%%%%%%%%%%%%%%%%%%%%% sec 5.1 %%%%%%%%%%%%%%%%%%%%%%%%%%%%%%%%%%%%%%%%%%%%%%%%%%%%%%%%%%%%%%%%%%%%%%%%%%%%%%%%%%%%%%%%%%%%%%%

\subsection{\kaishu 流体力学之 Euler 方程方式}

物理学有一些基本定律, 其中之一就是质量守恒定律(conservation of mass). 利用微分方程可以讲这定律用数学的式子表达出来, 透过方程式以精确的数学理论来研究可以帮助我们理解大自然并探索宇宙的奥妙. Euler 方程是由质量守恒与动量守恒, 还有能量守恒三大定律所组成. 我们先考虑无黏性的理想气体: 空间任意一区域 $ V $, 假设其密度为 $\rho \left( {\boldsymbol{x},t} \right)$, 其质量等于

\begin{equation}
M\left( t \right) = \iiint_V {\rho \left( {\boldsymbol{x},t} \right)dV} \qquad  (\text{\kaishu 质量 $ = $ 密度 $ \times $  体积})
\label{eq5.1.1}
\end{equation}

质量的变化等于质量对时间 $ t $ 的微分

\begin{equation}
\frac{{dM\left( t \right)}}{{dt}} = \iiint_V {\frac{{\partial \rho  (\boldsymbol{x},t)}}{{\partial t}}dV}
\label{eq5.1.2}
\end{equation}

另一方面我们可以计算从 $ V $ 的表面 $ \partial V = S$ 进出的质量得出质量的变率. 假设流体的速度是 $u\left( {\boldsymbol{x},t} \right)$ 则单位时间 $ dt $ 内流体移动的距离(长度)为  {\color{red} $ \boldsymbol{u} \times  dt $} 而高则为(投影) $ dt \ \boldsymbol{u}\cos \left(\boldsymbol{x}, \boldsymbol{n}\right) $ 其 $ \boldsymbol{n} $ 为法向量, 所以体积等于

\begin{equation}
\boldsymbol{u}\cos \left( {\boldsymbol{u},\boldsymbol{n}} \right)dtdS \qquad (\text{\kaishu 高 $ \times $ 底面积})
\end{equation}

其中 $ \cos \left( {\boldsymbol{u},\boldsymbol{n}} \right) $ 是 $ \boldsymbol{u},\boldsymbol{n} $ 的方向余弦. 此时的质量等于

\begin{equation}
\rho \boldsymbol{u}\cos \left( {\boldsymbol{u},\boldsymbol{n}} \right)dtdS \qquad (\text{\kaishu 密度 $ \times $ 体积})
\notag 
\end{equation}

沿着表面积分(曲面积分) 可得全部得质量

\begin{equation}
\iint_S {\rho \boldsymbol{u}\cos \left( {\boldsymbol{u},\boldsymbol{n}} \right)dtdS}
\label{eq5.1.4}
\end{equation}

这是时间 $ dt $ 内得质量, 所以外流的速度率(通量)为(除以 $ dt $)

\begin{equation}
\iint_S {\rho \boldsymbol{u}\cos \left( {\boldsymbol{u},\boldsymbol{n}} \right)dS} = \iint_S {\rho \boldsymbol{u} \cdot \boldsymbol{n}dS}
\notag 
\end{equation}

根据质量不灭(守恒)定律: 一物质的质量变化和该物质进出表面的改变(通量)是一样的. 因此

\begin{equation}
\iiint_V {\frac{{\partial \rho }}{{\partial t}}dV} =  - \iint_S {\rho \boldsymbol{u} \cdot \boldsymbol{n}dS}
\label{eq5.1.5}
\end{equation}

上面这个方程式右式的曲面积分可以运用散度定理转化成体积分(volume integral)

\begin{equation}
\iint_S {\rho \boldsymbol{u} \cdot \boldsymbol{n}dS} = \iiint_V {\nabla  \cdot \left( {\rho \boldsymbol{u}} \right)dV}
\notag
\end{equation}

因此

\begin{equation}
\iiint_V {\frac{{\partial \rho }}{{\partial t}} + \nabla  \cdot \left( {\rho \boldsymbol{u}} \right)dV} = 0
\label{eq5.1.6}
\end{equation}

由于我们考虑的是任意的区域 $ V $, 因此可以将 $ V $ 缩成只有一点再利用连续特质可结论: 被积分函数必定是 0

\begin{equation}
\frac{{\partial \rho }}{{\partial t}} + \nabla  \cdot \left( {\rho \boldsymbol{u}} \right) = 0
\label{eq5.1.7}
\end{equation}

或 $\boldsymbol{u} = \left( {u,v,w} \right)$

\begin{equation}
\frac{{\partial \rho }}{{\partial t}} + \frac{\partial }{{\partial x}}\left( {\rho u} \right) + \frac{\partial }{{\partial y}}\left( {\rho v} \right) + \frac{\partial }{{\partial z}}\left( {\rho w} \right) = 0
\label{eq5.1.8}
\end{equation}

这个微分方程 \ref{eq5.1.7} 或 \ref{eq5.1.8} 称为连续方程(continuity equation). 它就是质量守恒律: 流体沿别界 $\partial V = S$ 的流量多寡(通量)等于全部质量随时间的该变量. 

其次我们推导动量守恒律(conservation of momentum), 压力 $ P $ 作用到边界 $\partial V = S$ 水平方向 $ x $ 轴的力等于:

\begin{equation}
 - \iint_S {Pdydz} =  - \iiint_V {\frac{{\partial P}}{{\partial x}}dV} \qquad (\text{\color{cyan} \kaishu 力 = 压力 $ \times $ 面积 })
\label{eq5.19}
\end{equation}

另外位置向量为 $\boldsymbol{r} = \left( {x,y,z} \right) = \left( {x\left( t \right),y\left( t \right),z\left( t \right)} \right)$, 所以 $ x- $方向速度 $\mathop x\limits^ \cdot   = u$, 而 $ x- $方向加速度为 $\mathop x\limits^{ \cdot  \cdot }  = \mathop u\limits^ \cdot  $, 整体的加速度等于

\begin{equation}
\frac{{du}}{{dt}} = \frac{{\partial u}}{{\partial t}} + u\frac{{\partial u}}{{\partial x}} + v\frac{{\partial u}}{{\partial y}} + w\frac{{\partial u}}{{\partial z}} = \frac{{\partial u}}{{\partial t}} + \left( {\boldsymbol{u} \cdot \nabla } \right)u
\label{eq5.1.10}
\end{equation}

由牛顿第二运动定律 $ F = ma $, 可得

\begin{equation}
\frac{{du}}{{dt}} = \mathop {\lim }\limits_{\left| V \right| \to 0} \left( {\frac{{ - \iiint_V {\frac{{\partial P}}{{\partial x}}dV}}}{{\iiint_V {\rho dV}}}} \right)
\label{eq5.11}
\end{equation}

再次利用连续性:

\begin{equation}
\frac{{\partial u}}{{\partial t}} + u\frac{{\partial u}}{{\partial x}} + v\frac{{\partial u}}{{\partial y}} + w\frac{{\partial u}}{{\partial z}} + \frac{1}{\rho }\frac{{\partial P}}{{\partial x}} = 0
\label{eq5.1.12}
\end{equation}

同理如果考虑 $ y,z $ 方向:

\begin{equation}
\frac{{\partial v}}{{\partial t}} + u\frac{{\partial v}}{{\partial x}} + v\frac{{\partial v}}{{\partial y}} + w\frac{{\partial v}}{{\partial z}} + \frac{1}{\rho }\frac{{\partial P}}{{\partial y}} = 0
\label{eq5.1.13}
\end{equation}

\begin{equation}
\frac{{\partial w}}{{\partial t}} + u\frac{{\partial w}}{{\partial x}} + v\frac{{\partial w}}{{\partial y}} + w\frac{{\partial w}}{{\partial z}} + \frac{1}{\rho }\frac{{\partial P}}{{\partial z}} = 0
\label{eq5.1.14}
\end{equation}

或者将 \ref{eq5.1.12} -- \ref{eq5.1.14} 写成向量的形式:

\begin{equation}
{\partial _t}\boldsymbol{u} + \left( {\boldsymbol{u} \cdot \nabla } \right)\boldsymbol{u} + \frac{1}{\rho }\nabla P = 0
\label{eq5.1.15}
\end{equation}

这就是动量守恒律, 它本质上就是牛顿第二运动定律. 假设压力只与密度有关 $ P = P\left(\rho \right) $ 则 \ref{eq5.1.7}, \ref{eq.5.1.15} 就是著名的可压缩 Euler 方程(compressible Euler equation).

\begin{definition}[\kaishu 可压缩 Euler 方程]
	假设压力只与密度有关 $P = P\left( \rho  \right)$
	
	\begin{equation}
	\frac{{\partial \rho }}{{\partial t}}   + \nabla  \cdot \left( {\rho \boldsymbol{u}} \right)  = 0
	\notag 
	\end{equation}
	\vspace{-0.5cm}
	\begin{equation}
	{\partial _t}\boldsymbol{u} +  \left( {\boldsymbol{u} \cdot \nabla } \right)\boldsymbol{u} + \frac{1}{\rho }\nabla P  = 0
	\notag 
	\end{equation}
	
	如果有外力项 $ \boldsymbol{f} $ 则为
	
	\begin{equation}
	\frac{{\partial \rho }}{{\partial t}}   + \nabla  \cdot \left( {\rho \boldsymbol{u}} \right)  = 0
	\label{eq5.1.16}
	\end{equation}
	
	\vspace{-1cm}
	
	\begin{equation}
	{\partial _t}\boldsymbol{u} + \left( {\boldsymbol{u} \cdot \nabla } \right)\boldsymbol{u} + \frac{1}{\rho }\nabla P = \frac{\boldsymbol{f}}{\rho }
	\label{eq5.1.17}
	\end{equation}
	
\end{definition}

\begin{remark}
	\text{}
	\vspace{-0.5cm}
	\begin{enumerate}
		\item 任何一个有物理意义的方程必须满足量纲平衡(dimensional balance) 之要求, 质量守恒律 \ref{eq5.1.7}:
		\begin{equation}
		\left[ {\frac{{\partial \rho }}{{\partial t}}} \right] \approx \left[ {\nabla  \cdot \left( {\rho {\boldsymbol{u}}} \right)} \right] \Rightarrow \frac{1}{T}\frac{M}{{{L^3}}} = \frac{1}{L}\frac{M}{{{L^3}}}\frac{L}{T}
		\notag 
		\end{equation}
		\item 动量守恒律 \ref{eq5.1.17}:
		\begin{equation}
		\left[ {{\partial _t}\boldsymbol{u}} \right] \approx \left[ {\left( {\boldsymbol{u} \cdot \nabla } \right)\boldsymbol{u}} \right] \approx \left[ {\frac{1}{\rho }\nabla P} \right] \approx \left[ {\frac{\boldsymbol{f}}{\rho }} \right]
		\notag 
		\end{equation}
		\begin{equation}
		\frac{1}{T}\frac{L}{T} = \frac{L}{T}\frac{1}{L}\frac{L}{T} = \frac{1}{{{M \mathord{\left/
						{\vphantom {M {{L^3}}}} \right.
						\kern-\nulldelimiterspace} {{L^3}}}}}\frac{1}{L}\frac{{{{ML} \mathord{\left/
						{\vphantom {{ML} {{T^2}}}} \right.
						\kern-\nulldelimiterspace} {{T^2}}}}}{{{L^2}}} = \frac{{{{ML} \mathord{\left/
						{\vphantom {{ML} {{T^2}}}} \right.
						\kern-\nulldelimiterspace} {{T^2}}}}}{{{M \mathord{\left/
						{\vphantom {M {{L^3}}}} \right.
						\kern-\nulldelimiterspace} {{L^3}}}}}
		\notag 
		\end{equation}
		这里我们用到关系式 $P = {F \mathord{\left/
				{\vphantom {F A}} \right.
				\kern-\nulldelimiterspace} A},\;F = ma$, 因此若 $ P $ 是压力({\color{red} 压强}),  $ \boldsymbol{f} $ 代表力则 \ref{eq5.1.7} 的第三项,第四项必定含有 $\frac{1}{\rho }$ 这一项.
	    \item 利用连续方程 \ref{eq5.1.7},\ref{eq5.1.15}, \ref{eq5.1.17} 可以改写为
	    \begin{equation}
	    \color{red}
	    {\partial _t}\left( {\rho \boldsymbol{u}} \right) + div\left( {\rho \boldsymbol{u} \otimes \boldsymbol{u}} \right) + \nabla P = 0
	    \notag 
	    \end{equation}
	    \vspace{-1cm}
	    \begin{equation}
	    \color{red}
	    {\partial _t}\left( {\rho \boldsymbol{u}} \right) + div\left( {\rho \boldsymbol{u} \otimes \boldsymbol{u}} \right) + \nabla P = \boldsymbol{f}
	    \notag 
	    \end{equation}
	\end{enumerate}
\end{remark}

如果将黏性(viscosity)也考虑进来就是著名的 Navier-Stokes 方程

\begin{definition}[\kaishu 可压缩 Navier-Stokes 方程]
	$ P = P(\rho) $,
	\begin{equation}
	\frac{{\partial \rho }}{{\partial t}} + \nabla  \cdot \left( {\rho \boldsymbol{u}} \right) = 0
	\label{eq5.1.18}
	\end{equation}
	\begin{equation}
	{\partial _t}\boldsymbol{u} + \left( {\boldsymbol{u} \cdot \nabla } \right)\boldsymbol{u} + \frac{1}{\rho }\nabla P = \nu \Delta \boldsymbol{u} + \frac{\boldsymbol{f}}{\rho }
	\label{eq5.1.19}
	\end{equation}
	根据量纲平衡之要求, 黏性系数 $ \nu  $ 必须带有量纲 $\left[ v \right] = \frac{{{L^2}}}{T}$. 
\end{definition}

\begin{definition}[\kaishu 随物导数]
	我们称
	\begin{equation}
	\frac{D}{{Dt}} = \frac{\partial }{{\partial t}} + \boldsymbol{u} \cdot \nabla  = \frac{\partial }{{\partial t}} + u\frac{\partial }{{\partial x}} + v\frac{\partial }{{\partial y}} + w\frac{\partial }{{\partial z}}
	\label{eq5.1.20}
	\end{equation}
	这个算子为随物导数(material derivative), 它表示当我们随着一个流体观测它性质的变化率.
\end{definition}

要推导能量守恒律, 必须引进热力学的概念. 其中常见的有熵(entropy) S, 比热(ratio of specific heat) $ \gamma $, $ \gamma > 1 $. 此时压力({\color{red} 压强}),熵与密度之关系为:

\begin{equation}
P(\rho, S) = A(S) \rho^{\gamma}
\label{eq5.1.21}
\end{equation}

则能量守恒律其实就是熵沿粒子路径(particle path) 都是固定数的常数, 即:

\begin{equation}
\frac{{DS}}{{Dt}} = \frac{{\partial S}}{{\partial t}} + u\frac{{\partial S}}{{\partial x}} + v\frac{{\partial S}}{{\partial y}} + w\frac{{\partial S}}{{\partial z}} = \left( {\frac{\partial }{{\partial t}} + \boldsymbol{u} \cdot \nabla } \right)S = 0
\label{eq5.1.22}
\end{equation}

也就是说当我们跟定一个流体粒子时它的熵是不变的. 利用 \ref{eq5.1.21},\ref{eq5.1.22} 式也可以改写为:
\begin{equation}
\frac{\partial }{{\partial t}}\left( {\frac{P}{{{\rho ^\gamma }}}} \right) + u\frac{\partial }{{\partial x}}\left( {\frac{P}{{{\rho ^\gamma }}}} \right) + v\frac{\partial }{{\partial y}}\left( {\frac{P}{{{\rho ^\gamma }}}} \right) + w\frac{\partial }{{\partial z}}\left( {\frac{P}{{{\rho ^\gamma }}}} \right) = 0
\label{eq5.1.23}
\end{equation}

\vspace{-1cm}

\begin{equation}
\frac{D}{{Dt}}\left( {\frac{P}{{{\rho ^\gamma }}}} \right) = \frac{\partial }{{\partial t}}\left( {\frac{P}{{{\rho ^\gamma }}}} \right) + \left( {\boldsymbol{u} \cdot \nabla } \right)\left( {\frac{P}{{{\rho ^\gamma }}}} \right) = 0
\label{eq5.1.24}
\end{equation}

如果没有外力, ${\rho _0},{P_0}$ 是常数, 显然 $\left( {\boldsymbol{u},\rho ,P\left( \rho  \right)} \right) = \left( {0,{\rho _0},{P_0}} \right)$ 是 Euler 方程的解, 考虑小扰动

\begin{equation}
\boldsymbol{u} = \epsilon \boldsymbol{u}_{1}, \quad \rho = \rho_0 + \epsilon \rho_1, \quad P(\rho) = P_{0} + \epsilon P_{1} = P_{0} + \epsilon P^{\prime}(\rho_0)\rho_1
\label{eq5.1.25}
\end{equation}

代入 Euler 方程(我们承认 $ (\boldsymbol{u}, \rho, P(\rho)) $ 是 Euler 方程的解) 并忽略高阶项可得声学方程式 (Acoustic equation).

\begin{theorem}[\kaishu 声学方程式]
	\begin{equation}
	\color{red}
	\frac{{\partial {\rho _1}}}{{\partial t}} + {\rho _0}div\;{\boldsymbol{u}_1} = 0
	\label{eq5.1.26}
	\end{equation}
	
	\vspace{-1.5cm}
	
	\begin{equation}
	\color{red}
	\frac{{\partial {\boldsymbol{u}_1}}}{{\partial t}} + \frac{{P^{\prime}\left( {{\rho _0}} \right)}}{{{\rho _0}}}\nabla {\rho _1} = 0
	\label{eq5.1.27}
	\end{equation}
\end{theorem}

\begin{remark}
	
	\text{}
	
	\vspace{-0.75cm}
	
	\begin{enumerate}
		\item 声学方程式 \ref{eq5.1.26}, \ref{eq5.1.27} 也称为 Euler 方程之线性化(linearization). 这是研究非线性方程的最基本方法.
		\item 声学方程式 \ref{eq5.1.26}, \ref{eq5.1.27} 消去速度 $ \boldsymbol{u}_{1} $ 则密度 $ \rho_1 $ 满足波动方程
		\begin{equation}
		\frac{{{\partial ^2}{\rho _1}}}{{\partial {t^2}}} = {P^{\prime}}\left( {{\rho _0}} \right)\Delta {\rho _1} \equiv {c^2}\Delta {\rho _1}
		\label{eq5.1.28}
		\end{equation}
		$c = \sqrt {{P^{\prime}}\left( {{\rho _0}} \right)} $ 是声速(sound speed). 为什么 $c = \sqrt {{P^{\prime}}\left( {{\rho _0}} \right)} $  是声速? 籍由量纲分析
		\begin{equation}
		{\left[ c \right]^2} = \left[ {{P^{\prime}}\left( {{\rho _0}} \right)} \right] = \frac{{\left[ P \right]}}{{\left[ {{\rho _0}} \right]}} = \frac{{{M \mathord{\left/
						{\vphantom {M {L{T^2}}}} \right.
						\kern-\nulldelimiterspace} {L{T^2}}}}}{{{M \mathord{\left/
						{\vphantom {M {{L^3}}}} \right.
						\kern-\nulldelimiterspace} {{L^3}}}}} = \frac{{{L^2}}}{{{T^2}}}
		\notag
		\end{equation}
		$ c $ 之量纲(dimension) 正是速度! 压力({\color{red} 压强}) 对密度的微分是声速的平方. 
		\item 波动方程 \ref{eq5.1.28} 是线性而压力 $P = P\left( {{\rho _0}} \right)$ 是密度 $ \rho_0 $ 的函数因此则压力 $ P $ 也满足相同的波动方程
		\begin{equation}
		\frac{{{\partial ^2}P}}{{\partial {t^2}}} = {c^2}\Delta P
		\label{eq5.1.29}
		\end{equation}
		\item 如果假设 $\nabla  \times {\boldsymbol{u}_1} = 0$ 则
		\begin{equation}
		\nabla div\;{\boldsymbol{u}_1}\; = \Delta {\boldsymbol{u}_1} + \nabla  \times \left( {\nabla  \times {\boldsymbol{u}_1}} \right) = \Delta {\boldsymbol{u}_1}
		\notag 
		\end{equation}
		消去密度 $ \rho_1 $ 则速度 $ \boldsymbol{u}_{1} $ 也满足波动方程
		
		\begin{equation}
		\frac{{{\partial ^2}{\boldsymbol{u}_1}}}{{\partial {t^2}}} = {c^2}\Delta {\boldsymbol{u}_1}
		\label{eq5.1.30}
		\end{equation}
		
		若引进速度位 $ U $, $\nabla U = {\boldsymbol{u}_1}$ 则 $ U $ 也满足相同的波动方程
		
		\begin{equation}
		\frac{{{\partial ^2}U}}{{\partial {t^2}}} = {c^2}\Delta U
		\label{eq5.1.31}
		\end{equation}
		
		因此声学方程式中的密度 $ \rho_1 $, 压力 $ P_{1} $, 速度 $ \boldsymbol{u}_{1} $ 与速度位 $ U $ 都满足相同的波动方程且以相同的速率 $ \pm c$ 传递.
	\end{enumerate}
\end{remark}


\begin{homework}
	\text{}
	
	\vspace{-0.5cm}
	
	\begin{enumerate}
		\item 圆柱坐标之速度 $\boldsymbol{u} = {u_r}{\boldsymbol{e}_r} + {u_\theta }{\boldsymbol{e}_\theta } + {u_z}{\boldsymbol{e}_z}$, 证明连续方程 \ref{eq5.1.8} 可以改写为
		\begin{equation}
		\frac{{\partial \rho }}{{\partial t}} + \frac{1}{r}\frac{\partial }{{\partial r}}\left( {r\rho {u_r}} \right) + \frac{1}{r}\frac{\partial }{{\partial \theta }}\left( {\rho {u_\theta }} \right) + \frac{\partial }{{\partial z}}\left( {\rho {u_z}} \right) = 0
		\notag 
		\end{equation}
		\item 球坐标之速度 $\boldsymbol{u} = u{}_r{\boldsymbol{e}_r} + {u_\varphi }{\boldsymbol{e}_\varphi } + {u_\theta }{\boldsymbol{e}_\theta }$, 证明连续方程 \ref{eq5.1.8} 可以改写为
		\begin{equation}
		\frac{{\partial \rho }}{{\partial t}} + \frac{1}{{{r^2}}}\frac{\partial }{{\partial r}}\left( {{r^2}\rho {u_r}} \right) + \frac{1}{{r\sin \varphi }}\frac{\partial }{{\partial \varphi }}\left( {\sin \varphi \rho {u_\varphi }} \right) + \frac{1}{{r\sin \varphi }}\frac{\partial }{{\partial \theta }}\left( {\rho {u_\theta }} \right) = 0
		\notag
		\end{equation}
		(这里我们以 $r = \sqrt {{x^2} + {y^2} + {z^2}} $ 代替 $ \rho $, 免得与密度 $ \rho $ 相混肴! )
	\end{enumerate}
\label{hom5.1}
\end{homework}


%%%%%%%%%%%%%%%%%%%%%%%%%%%%%%%%%%%%%%%%%%%%%%%%%%%%%%%%%%%%%%%%%% sec 5.2 %%%%%%%%%%%%%%%%%%%%%%%%%%%%%%%%%%%%%%%%%%%%%%%%%%%%%%%%%%%%%%%%%%%%%%%%%%%%%%%%%%%%%%%%%%%%%%%

\subsection{\kaishu 不可压缩流体}

有了速度的概念之后, 我们觉得这是很好的时机来阐释散度(divergence) 的物理意义, 我们可以这么想象: 假设正在喝咖啡将奶精倒入杯内, 最初形成的图形为 $ \Omega_{0} $, 而后经由搅拌或其他因素使得$ \Omega_{0} $ 变化, 其位置向量为 $\left( {x\left( t \right),y\left( t \right)} \right) = \left( {\xi ,\eta } \right)$, 速度则为 $ \boldsymbol{u}\left(t\right)  = \left( {\frac{{dx}}{{dt}},\frac{{dy}}{{dt}}} \right) = \left( {u,v} \right)$, 经过 $ t $ 时间之后 $ \Omega_{0} $ 转换为 $ \Omega_{t} $, 我们有兴趣的问题是 $ \Omega_{0} $ 与 $ \Omega_{t} $, 两者之面积变化为何? (实际上牛顿力学本身就是一种坐标变换) 我们看其中一小块($d{V_0} \to d{V_t}$) 两者关系为

\begin{equation}
d{V_0} = dxdy = \frac{{\partial \left( {x,y} \right)}}{{\partial \left( {\xi ,\eta } \right)}} = J\left( t \right)d\xi d\eta  = J\left( t \right)d{V_t}
\label{eq5.2.1}
\end{equation}

其中 $J\left( t \right)$ 就是 Jacobian

\begin{equation}
J\left( t \right) = \left| {\frac{{\partial \left( {x,y} \right)}}{{\partial \left( {\xi ,\eta } \right)}}} \right| = \left| {\begin{matrix}
	{\frac{{\partial x}}{{\partial \xi }}} & {\frac{{\partial x}}{{\partial \eta }}}  \\ 
	{\frac{{\partial y}}{{\partial \xi }}} & {\frac{{\partial y}}{{\partial \eta }}}  \\ 
	\end{matrix} } \right| = \frac{{d{V_0}}}{{d{V_t}}}
\label{eq5.2.2}
\end{equation}

\begin{figure}[!htb]
	\centering
	\includegraphics[width=0.5\linewidth,height=0.25\linewidth]{fig_5_2_1.png}
	\label{img001}
	\caption{}
\end{figure}

因此 ${\Omega _0} \to {\Omega _t}$ 之变化情形, 等于是探讨 $ J\left(t\right) $ 的变化, 我们假设 $ J,J^{-1} \ne 0 $, 即没有退化的情形(可设 $ 0 < J < \infty $ ): 由全微分(total differential) 知:

\begin{equation}
\frac{d}{{dt}}\left( {\frac{{\partial x}}{{\partial \xi }}} \right) = \frac{\partial }{{\partial \xi }}\left( {\frac{{dx}}{{dt}}} \right) = \frac{{\partial u}}{{\partial \xi }} = \frac{{\partial u}}{{\partial x}}\frac{{\partial x}}{{\partial \xi }} + \frac{{\partial u}}{{\partial y}}\frac{{\partial y}}{{\partial \xi }}
\notag 
\end{equation}

同理可得

\begin{equation}
\begin{split}
\frac{d}{{dt}}\left( {\frac{{\partial x}}{{\partial \eta }}} \right) & = \frac{{\partial u}}{{\partial x}}\frac{{\partial x}}{{\partial \eta }} + \frac{{\partial u}}{{\partial y}}\frac{{\partial y}}{{\partial \eta }}\\
\frac{d}{{dt}}\left( {\frac{{\partial y}}{{\partial \xi }}} \right) & = \frac{{\partial u}}{{\partial x}}\frac{{\partial x}}{{\partial \xi }} + \frac{{\partial u}}{{\partial y}}\frac{{\partial y}}{{\partial \xi }} \\
\frac{d}{{dt}}\left( {\frac{{\partial y}}{{\partial \eta }}} \right) & = \frac{{\partial u}}{{\partial x}}\frac{{\partial x}}{{\partial \eta }} + \frac{{\partial u}}{{\partial y}}\frac{{\partial y}}{{\partial \eta }}
\end{split}
\notag 
\end{equation}

$ J\left(t\right) $ 对 $ t $ 微分并利用行列式之性质可得
\begin{equation}
\begin{split}
\frac{{dJ}}{{dt}} & = \left| {\begin{matrix}
	{\frac{d}{{dt}}\frac{{\partial x}}{{\partial \xi }}} & {\frac{d}{{dt}}\frac{{\partial x}}{{\partial \eta }}}  \\ 
	{\frac{{\partial y}}{{\partial \xi }}} & {\frac{{\partial y}}{{\partial \eta }}}  \\ 
	\end{matrix} } \right| + \left| {\begin{matrix}
	{\frac{{\partial x}}{{\partial \xi }}} & {\frac{{\partial x}}{{\partial \eta }}}  \\ 
	{\frac{d}{{dt}}\frac{{\partial y}}{{\partial \xi }}} & {\frac{d}{{dt}}\frac{{\partial y}}{{\partial \eta }}}  \\ 
	\end{matrix} } \right| \\
				 & = \left| {\begin{matrix}
				 	{\frac{{\partial u}}{{\partial x}}\frac{{\partial x}}{{\partial \xi }} + \frac{{\partial u}}{{\partial y}}\frac{{\partial y}}{{\partial \xi }}} & {\frac{{\partial u}}{{\partial x}}\frac{{\partial x}}{{\partial \eta }} + \frac{{\partial u}}{{\partial y}}\frac{{\partial y}}{{\partial \eta }}}  \\ 
				 	{\frac{{\partial y}}{{\partial \xi }}} & {\frac{{\partial y}}{{\partial \eta }}}  \\ 
				 	\end{matrix} } \right| + \left| {\begin{matrix}
				 	{\frac{{\partial x}}{{\partial \xi }}} & {\frac{{\partial x}}{{\partial \eta }}}  \\ 
				 	{\frac{{\partial v}}{{\partial x}}\frac{{\partial x}}{{\partial \xi }} + \frac{{\partial v}}{{\partial y}}\frac{{\partial y}}{{\partial \xi }}} & {\frac{{\partial v}}{{\partial x}}\frac{{\partial x}}{{\partial \eta }} + \frac{{\partial v}}{{\partial y}}\frac{{\partial y}}{{\partial \eta }}}  \\ 
				 	\end{matrix} } \right|\\
			 	& = \left| {\begin{matrix}
			 		{\frac{{\partial u}}{{\partial x}}\frac{{\partial x}}{{\partial \xi }}} & {\frac{{\partial u}}{{\partial x}}\frac{{\partial x}}{{\partial \eta }}}  \\ 
			 		{\frac{{\partial y}}{{\partial \xi }}} & {\frac{{\partial y}}{{\partial \eta }}}  \\ 
			 		\end{matrix} } \right| + \left| {\begin{matrix}
			 		{\frac{{\partial x}}{{\partial \xi }}} & {\frac{{\partial x}}{{\partial \eta }}}  \\ 
			 		{\frac{{\partial v}}{{\partial y}}\frac{{\partial y}}{{\partial \xi }}} & {\frac{{\partial v}}{{\partial y}}\frac{{\partial y}}{{\partial \eta }}}  \\ 
			 		\end{matrix} } \right|\\
		 		& = \frac{{\partial u}}{{\partial x}}\left| {\begin{matrix}
		 			{\frac{{\partial x}}{{\partial \xi }}} & {\frac{{\partial x}}{{\partial \eta }}}  \\ 
		 			{\frac{{\partial y}}{{\partial \xi }}} & {\frac{{\partial y}}{{\partial \eta }}}  \\ 
		 			\end{matrix} } \right| + \frac{{\partial v}}{{\partial y}}\left| {\begin{matrix}
		 			{\frac{{\partial x}}{{\partial \xi }}} & {\frac{{\partial x}}{{\partial \eta }}}  \\ 
		 			{\frac{{\partial y}}{{\partial \xi }}} & {\frac{{\partial y}}{{\partial \eta }}}  \\ 
		 			\end{matrix} } \right|\\
	 			& = \left( {\frac{{\partial u}}{{\partial x}} + \frac{{\partial v}}{{\partial y}}} \right)\left| {\begin{matrix}
	 				{\frac{{\partial x}}{{\partial \xi }}} & {\frac{{\partial x}}{{\partial \eta }}}  \\ 
	 				{\frac{{\partial y}}{{\partial \xi }}} & {\frac{{\partial y}}{{\partial \eta }}}  \\ 
	 				\end{matrix} } \right|\\
 				& = \left( {\frac{{\partial u}}{{\partial x}} + \frac{{\partial v}}{{\partial y}}} \right)J\left( t \right)\\
 				& = \nabla  \cdot \boldsymbol{u} \ J\left( t \right)
\end{split}
\notag
\end{equation}

所以 $ J\left(t\right) $ 满足一阶微分方程:
\begin{equation}
\frac{{dJ}}{{dt}} = \left( {\nabla  \cdot \boldsymbol{u}} \right)J
\notag 
\end{equation}

前面之推导对于三个变数 $\left( {x,y,z} \right)$ 仍然成立, 此时

\begin{equation}
\boldsymbol{u} = \left(u,v,w\right), \quad J\left( t \right) = \left| {\frac{{\partial \left( {x,y,z} \right)}}{{\partial \left( {\xi ,\eta ,\zeta } \right)}}} \right| = \left| {\begin{matrix}
	{\frac{{\partial x}}{{\partial \xi }}} & {\frac{{\partial x}}{{\partial \eta }}} & {\frac{{\partial x}}{{\partial \zeta }}}  \\ 
	{\frac{{\partial y}}{{\partial \xi }}} & {\frac{{\partial y}}{{\partial \eta }}} & {\frac{{\partial y}}{{\partial \zeta }}}  \\ 
	{\frac{{\partial z}}{{\partial \xi }}} & {\frac{{\partial z}}{{\partial \eta }}} & {\frac{{\partial z}}{{\partial \zeta }}}  \\ 
	\end{matrix} } \right|
\notag 
\end{equation}

在 $ n $ 维空间这公式仍然成立, 我们称为 Euler 展开公式.

\begin{theorem}[\kaishu Euler 展开公式]
	$ J\left(t\right) $ 满足微分方程
	\begin{equation}
	\frac{{dJ}}{{dt}} = \left( {\nabla  \cdot \boldsymbol{u}} \right)J \quad \Leftrightarrow \quad  \frac{{d\;\ln \;J}}{{dt}} = \nabla  \cdot \boldsymbol{u} \quad \Leftrightarrow \quad  J\left( t \right) = {e^{t\left( {\nabla  \cdot \boldsymbol{u}} \right)}}J\left( 0 \right)
	\label{eq5.2.3}
	\end{equation}
\end{theorem}

由 Euler 展开公式 \ref{eq5.2.3} 可知速度场 $ \boldsymbol{u} $ 之散度的物理或几何意义如下:

\vspace{-0.75cm}

\begin{equation}
\begin{split}
\nabla  \cdot u > 0\; \Leftrightarrow \;J\left( t \right) > J\left( 0 \right)\; \Leftrightarrow \; \text{\kaishu 面积变大} \\
\nabla  \cdot u =  0\; \Leftrightarrow \;J\left( t \right) > J\left( 0 \right)\; \Leftrightarrow \; \text{\kaishu 面积不变}\\
\nabla  \cdot u <  0\; \Leftrightarrow \;J\left( t \right) > J\left( 0 \right)\; \Leftrightarrow \; \text{\kaishu 面积变小}
\end{split}
\label{eq5.2.4}
\end{equation}



因此我们称一流体为不可压缩(incompressible) 其真实意义:

流体经过任何的变换其形状虽然改变了, 但其面积(或体积)欲始终保持不变. 

\begin{theorem}[\kaishu Reynolds 传输定理 \uppercase\expandafter{\romannumeral1}]
	$f = f\left( {\boldsymbol{x},t} \right)$ 是可微函数, 则
	\begin{equation}
	\frac{d}{{dt}}\iiint_{V\left( t \right)} {f\left( {\boldsymbol{x},t} \right)dV} = \iiint_{V\left( t \right)} {\frac{{Df}}{{Dt}}dV = \iiint_{V\left( t \right)} {\left( {\frac{{\partial f}}{{\partial t}} + div\left( {f\boldsymbol{u}} \right)} \right)dV}}
	\label{eq5.2.5}
	\end{equation}
\end{theorem}

\begin{proof}
	\kaishu 我们考虑变数变换, $\boldsymbol{x} = \boldsymbol{x}\left( {\xi ,t} \right),\;dV\left( t \right) = Jd{V_0}$ 则
	\begin{equation}
	\begin{split}
	\frac{d}{{dt}}\iiint_{V\left( t \right)} {f\left( {\boldsymbol{x},t} \right)dV} & = \frac{d}{{dt}}\iiint_{{V_0}} {f\left( {\boldsymbol{x}\left( {\xi ,t} \right)} \right)Jd{V_0}}\\
																					& = \iiint_{{V_0}} {\left( {\frac{{df}}{{dt}}J + f\frac{{dJ}}{{dt}}} \right)d{V_0}}\\
																					& = \iiint_{{V_0}} {\left( {\frac{{df}}{{dt}} + f\frac{{dJ}}{{Jdt}}} \right)Jd{V_0}}\\
																					& = \iiint_{{V_0}} {\left( {\frac{{df}}{{dt}} + f\left( {div\;\boldsymbol{u}} \right)} \right)Jd{V_0}}\\
																					& = \iiint_V {\left( {\frac{{df}}{{dt}} + \ div\;(f\boldsymbol{u})} \right) dV}
	\end{split}
	\notag 
	\end{equation}
	
	但 $\frac{D}{{Dt}} = \frac{\partial }{{\partial t}} + \boldsymbol{u} \cdot \nabla $, 所以
	
	\begin{equation}
	\color{red}
	\begin{split}
	\frac{d}{{dt}}\iiint_{V\left( t \right)} {f\left( {\boldsymbol{x},t} \right)dV}  = \iiint_{V\left( t \right)} {\left( {\frac{{\partial f}}{{\partial t}} + div\left( {f\boldsymbol{u}} \right)} \right)dV} = \iiint_{V\left( t \right)} {\frac{{\partial f}}{{\partial t}}dV} + \iiint_{S\left( t \right)} {f\left( {\boldsymbol{x},t} \right)\boldsymbol{u} \cdot \boldsymbol{n}dS}
	\end{split}
	\notag 
	\end{equation}
	
\end{proof}

\newpage 

如果把密度 $ \rho $ 也考虑进来, 则由连续方程可得以下之传输定理.


\begin{theorem}[\kaishu Reynolds 传输定理 \uppercase\expandafter{\romannumeral2}]
	\begin{equation}
	\frac{d}{{dt}}\iiint_{V\left( t \right)} {\rho \left( {\boldsymbol{x},t} \right)f\left( {\boldsymbol{x},t} \right)dV} = \iiint_{V\left( t \right)} {\frac{{Df}}{{Dt}}dV} = \iiint_{V\left( t \right)} {\rho \left( {\boldsymbol{x},t} \right)\frac{{Df}}{{Dt}}dV}
	\label{eq5.2.6}
	\end{equation}
\end{theorem}

\begin{example}
	令 ${x^2} + {y^2} + {z^2} = {r^2},\;{u^2} + {v^2} + {w^2} = \rho ,\;$ 证明
	
	\begin{equation}
	\frac{d}{{dt}}\iiint_{{r^2} \leqslant {t^2}} {f\left( {x,y,z,t} \right)dxdydz} = \iint_{{r^2} = {t^2}} {f\left( {x,y,z,t} \right)dS} + \iiint_{{r^2} \leqslant {t^2}} {\frac{{\partial f}}{{\partial t}}dxdydz}
	\label{eq5.2.7}
	\end{equation}
\end{example}

\begin{proof}
	
	\kaishu 
	
	\begin{equation}
	\color{magenta}
	\underbrace {{{\left( {\frac{x}{t}} \right)}^2}}_{{u^2}} + \underbrace {{{\left( {\frac{y}{t}} \right)}^2}}_{{v^2}} + \underbrace {{{\left( {\frac{z}{t}} \right)}^2}}_{{w^2}} = \frac{{\left( {{x^2} + {y^2} + {z^2}} \right)}}{{{t^2}}} = \frac{{{r^2}}}{{{t^2}}} = \rho  \Rightarrow {x^2} + {y^2} + {z^2} = {t^2}\rho 
	\notag 
	\end{equation}
	
	\vspace{-1cm}
	
	\begin{equation}
	\color{magenta}
	{x^2} + {y^2} + {z^2} = {t^2}\rho  \Rightarrow {x^2} + {y^2} + {z^2} \leqslant {t^2} \Leftrightarrow {t^2}\rho  \leqslant {t^2} \Leftrightarrow \rho  \leqslant 1\;\;\;\left( {{t^2} > 0} \right)
	\notag 
	\end{equation}
	
	这公式 \ref{eq5.2.7} 可视为微积分基本定理之推广
	
	\begin{equation}
	\begin{split}
			& \frac{d}{{dt}}\iiint_{{r^2} \leqslant {t^2}} {f\left( {x,y,z,t} \right)dxdydz}\\
		=   & \frac{d}{{dt}}\iiint_{{\rho ^2} \leqslant 1} {{t^3}\left( {tu,tv,tw,t} \right)dudvdw} \\
		=   & \frac{d}{{dt}}\iiint_{{\rho ^2} \leqslant 1} {\left[ {{t^3}\left( {\frac{{\partial f}}{{\partial x}}u + \frac{{\partial f}}{{\partial y}}v + \frac{{\partial f}}{{\partial z}}w + \frac{{\partial f}}{{\partial t}}} \right) + 3{t^2}f} \right]dudvdw} \\
		=   & \frac{1}{t}\iiint_{{r^2} \leqslant {t^2}} {\nabla  \cdot \left( {fx,fy,fz} \right)dxdydz} + \iiint_{{r^2} \leqslant {t^2}} {\frac{{\partial f}}{{\partial t}}dxdydz}\\
		=   & \frac{1}{t}\iint_{{r^2} = {t^2}} {\left( {fx,fy,fz} \right) \cdot \boldsymbol{n}dS} + \iiint_{{r^2} \leqslant {t^2}} {\frac{{\partial f}}{{\partial t}}dxdydz}
	\end{split}
	\notag 
	\end{equation}
	
	但是 $\boldsymbol{n} = \left( {\cos \alpha ,\cos \beta ,\cos \gamma } \right) = \left( {\frac{x}{t},\frac{y}{t},\frac{z}{t}} \right)$, 故
	
	\begin{equation}
	\frac{1}{t}\iint_{{r^2} = {t^2}} {\left( {fx,fy,fz} \right) \cdot ndS} = \iint_{{r^2} = {t^2}} {f\frac{{{x^2} + {y^2} + {z^2}}}{{{t^2}}}dS} = \iint_{{r^2} = {t^2}} {fdS}
	\notag
	\end{equation}
	
	另外直观上可如此看, 令
	
	\begin{equation}
	F\left( t \right) = \iiint_{{r^2} \leqslant {t^2}} {f\left( {x,y,z,t} \right)dxdydz}
	\notag
	\end{equation}
	
	则
	
	\begin{equation}
	\begin{split}
	F\left( {t + {t_0}} \right) - F\left( t \right) & = \iiint_{{r^2} \leqslant {{\left( {t + {t_0}} \right)}^2}} {f\left( {x,y,z,t + {t_0}} \right)dxdydz} - \iiint_{{r^2} \leqslant {t^2}} {f\left( {x,y,z,t} \right)dxdydz}\\
													& = \iiint_{{t^2} \leqslant {r^2} \leqslant {{\left( {t + {t_0}} \right)}^2}} {f\left( {x,y,z,t + {t_0}} \right)dxdydz} + \iiint_{{r^2} \leqslant {t^2}} {\left[ {f\left( {x,y,z,t + {t_0}} \right) - f\left( {x,y,z,t} \right)} \right]dxdydz}
	\end{split}
	\notag 
	\end{equation}
	
	因此
	
	\begin{equation}
	\begin{split}
	\frac{{dF}}{{dt}} & = \mathop {\lim }\limits_{{t_0} \to 0} \frac{{F\left( {t + {t_0}} \right) - F\left( t \right)}}{{{t_0}}}\\
					  & = \mathop {\lim }\limits_{{t_0} \to 0} \left[ {\frac{1}{{{t_0}}}\iiint_{{t^2} \leqslant {r^2} \leqslant {{\left( {t + {t_0}} \right)}^2}} {f\left( {x,y,z,t + {t_0}} \right)dxdydz} + \iiint_{{r^2} \leqslant {t^2}} {\frac{{f\left( {x,y,z,t + {t_0}} \right) - f\left( {x,y,z} \right)}}{{{t_0}}}dxdydz}} \right]
	\end{split}
	\notag 
	\end{equation}
	
	显然
	
	\begin{equation}
	\iiint_{{r^2} \leqslant {t^2}} {\frac{{f\left( {x,y,z,t + {t_0}} \right) - f\left( {x,y,z,t} \right)}}{{{t_0}}}dxdydz} \to \iiint_{{r^2} \leqslant {t^2}} {\frac{{\partial f}}{{\partial t}}\left( {x,y,z,t} \right)dxdydz}
	\notag 
	\end{equation}
	
	而体积的变化就是表面积, 因此
	
	\begin{equation}
	\frac{1}{{{t_0}}}\iiint_{{t^2} \leqslant {r^2} \leqslant {{\left( {t + {t_0}} \right)}^2}} {f\left( {x,y,z,t + {t_0}} \right)dxdydz} \to \iint_{{r^2} = {t^2}} {f\left( {x,y,z,t} \right)dS}
	\notag 
	\end{equation}
	
	
\end{proof}

Euler 展开公式 \ref{eq5.2.1}, 可以帮助我们了解何谓不可压缩性(incompressibility). 所谓不可压缩流体, 是指流体的体积不随时间改变

\begin{equation}
\left| {V\left( t \right)} \right| = \iiint_{V\left( t \right)} {dV} = \text{\kaishu 常数}
\notag 
\end{equation}

因此不可压缩性等价于

\begin{equation}
%\color{red}
0 = \frac{d}{{dt}}\iiint_{V\left( t \right)} {dV} = \frac{d}{{dt}}\iiint_{V\left( t \right)} {Jd{V_0}} = \iiint_{V\left( t \right)} {\left( {div\;\boldsymbol{u}} \right)Jd{V_0}} = \iiint_{V\left( t \right)} {\left( {div\;\boldsymbol{u}} \right)dV}
\notag 
\end{equation}

所以可以结论

\begin{theorem}
	
	底下之叙述是等价:
	
	\begin{enumerate}
		\item 流体为不可压缩
		\item $ div \ \boldsymbol{u} = \nabla \cdot \boldsymbol{u} = 0 $
		\item $ J = 1 $
	\end{enumerate}
	
\end{theorem}

由连续方程

\begin{equation}
\frac{{\partial \rho }}{{\partial t}} + div\;\left( {\rho \boldsymbol{u}} \right) = 0,\;\frac{{D\rho }}{{Dt}} + \rho \;div\;\boldsymbol{u}\; = 0
\notag 
\end{equation}

因为 $ \rho > 0 $, 所以流体是不可压缩的意思是等价于 $\frac{{D\rho }}{{Dt}}$, 意即流体之密度沿着流体流动方向是常数. 如果流体是均匀的(homogeneous), 意即流体密度 $ \rho  $ 在任何地方欲是一样(是一固定常数), 则流体是不可压缩的条件是等价于 $ div \ \boldsymbol{u} = 0 $.

\begin{definition}[\kaishu 不可压缩流体方程]
	\text{}
	
	\vspace{-0.5cm}
	
	\begin{enumerate}
		\item 不可压缩 Euler 方程
		\begin{equation}
		\frac{{\partial \boldsymbol{u}}}{{\partial t}} + \boldsymbol{u} \cdot \nabla \boldsymbol{u} + \nabla P = 0,\;\;\;div\;\boldsymbol{u} = 0
		\label{eq5.2.8}
		\end{equation}
		\item 不可压缩 Navier-Stokes 方程
		\begin{equation}
		\frac{{\partial \boldsymbol{u}}}{{\partial t}} + \boldsymbol{u} \cdot \nabla \boldsymbol{u} + \nabla P = \nu \Delta \boldsymbol{u},\;\;\;div\;\boldsymbol{u} = 0
		\label{eq5.2.9}
		\end{equation}
	\end{enumerate}
\end{definition}

\begin{homework}
	\text{}
	
	\vspace{-0.5cm}
	
	\begin{enumerate}
		\item 圆柱坐标之速度 $ \boldsymbol{u} = u_{r}\boldsymbol{e}_{r} + u_{\theta}\boldsymbol{e}_{\theta} + u_{z}\boldsymbol{e}_{z} $, 证明不可压缩 Navier-Stokes 方程 \ref{eq5.2.9} 可以改写为
		
		\begin{equation}
		\begin{split}
		\frac{{\partial {u_r}}}{{\partial t}} + \left( {\boldsymbol{u} \cdot \nabla } \right){u_r} - \frac{{u_\theta ^2}}{r} + \frac{1}{\rho }\frac{{\partial P}}{{\partial r}} & = \nu \left( {\Delta {u_r} - \frac{{{u_r}}}{{{r^2}}} - \frac{2}{{{r^2}}}\frac{{\partial {u_\theta }}}{{\partial \theta }}} \right)\\
		\frac{{\partial {u_\theta }}}{{\partial t}} + \left( {\boldsymbol{u} \cdot \nabla } \right){u_\theta } + \frac{{{u_r}{u_\theta }}}{r} + \frac{1}{{\rho r}}\frac{{\partial P}}{{\partial \theta }} & = \nu \left( {\Delta {u_\theta } + \frac{2}{{{r^2}}}\frac{{\partial {u_r}}}{{\partial \theta }} - \frac{{{u_\theta }}}{{{r^2}}}} \right)\\
		\frac{{\partial {u_z}}}{{\partial t}} + \left( {\boldsymbol{u} \cdot \nabla } \right){u_z} + \frac{1}{\rho }\frac{{\partial P}}{{\partial z}} & = \nu \Delta {u_z}\\
		\frac{1}{r}\frac{\partial }{{\partial r}}\left( {r{u_r}} \right) + \frac{1}{r}\frac{{\partial {u_\theta }}}{{\partial \theta }} + \frac{{\partial {u_z}}}{{\partial z}} & = 0
		\end{split}
		\notag 
		\end{equation}
		
		其中 
		\begin{equation}
		\begin{split}
		\left( {\boldsymbol{u} \cdot \nabla } \right) & = {u_r}\frac{\partial }{{\partial r}} + \frac{{{u_\theta }}}{r}\frac{\partial }{{\partial \theta }} + {u_z}\frac{\partial }{{\partial z}}\\
		\Delta  & = \frac{1}{r}\frac{\partial }{{\partial r}}\left( {r\frac{\partial }{{\partial r}}} \right) + \frac{1}{{{r^2}}}\frac{{{\partial ^2}}}{{\partial {\theta ^2}}} + \frac{{{\partial ^2}}}{{\partial {z^2}}}
		\end{split}
		\notag 
		\end{equation}
		
		速度 $ \boldsymbol{u} $ 每个分量之量纲为 $\left[ u \right] = \left[ {{u_r}} \right] = \left[ {{u_\theta }} \right] = \left[ {{u_z}} \right] = \frac{L}{T}$, 请验证每项都满足量纲平衡之关系. 
		\item 证明
		
		\begin{equation}
		\frac{d}{{dt}}\iint_S {\boldsymbol{a} \cdot \boldsymbol{n}\;dS} = \iint_S {\left( {\frac{{d\boldsymbol{a}}}{{dt}} + \boldsymbol{a}\left( {\nabla  \cdot \boldsymbol{u}} \right) - \left( {\boldsymbol{a} \cdot \nabla } \right)\boldsymbol{u}} \right) \cdot \boldsymbol{n}dS}
		\notag 
		\end{equation}
		
	\end{enumerate}
	\label{hom5.2}
\end{homework}

%%%%%%%%%%%%%%%%%%%%%%%%%%%%%%%%%%%%%%%%%%%%%%%%%%%%%%%%%%%%%%%%%% sec 5.3 %%%%%%%%%%%%%%%%%%%%%%%%%%%%%%%%%%%%%%%%%%%%%%%%%%%%%%%%%%%%%%%%%%%%%%%%%%%%%%%%%%%%%%%%%%%%%%%

\newpage 

\subsection{\kaishu 涡度与涡度方程}

可压缩流体主要研究的问题是震波(shock wave) 与 vortex, 但不可压缩流体的主要问题是 vortex, 因此需要引进涡度(vorticity) 这个概念.

\begin{definition}[\kaishu 涡度]
	已知流体之速度场 $\boldsymbol{u} = \left( {u,v,w} \right)$, 则其涡度(vorticity) 定义为
	
	\begin{equation}
	\boldsymbol{\omega} = \nabla \times \boldsymbol{u} = \left| {\begin{matrix}
		\boldsymbol{i} & \boldsymbol{j} & \boldsymbol{k}  \\ 
		{{\partial _x}} & {{\partial _y}} & {{\partial _z}}  \\ 
		u & v & w  \\ 
		\end{matrix} } \right| = \left( {{\partial _y}w - {\partial _z}v,{\partial _z}u - {\partial _x}w,{\partial _x}v - {\partial _y}u} \right)
	\label{eq5.3.1}
	\end{equation}
	
	它代表流体粒子瞬时转动角速度.
\label{def5.3.1}	
\end{definition}

直接由定义清楚可知涡度 $ \boldsymbol{\omega} $ 之量纲 $\left[ \boldsymbol{\omega}  \right] = \frac{{\left[ \boldsymbol{u} \right]}}{L} = \frac{1}{T}$.

\begin{theorem}
	流体之速度场 $ \boldsymbol{u} $ 可以表示为平移(translation), 形变(deformation) 还有旋转(rotation) 三部分
	
	\begin{equation}
	\boldsymbol{u}\left( \boldsymbol{y} \right) = \boldsymbol{u}\left( \boldsymbol{x} \right) + \mathcal{D}\left( \boldsymbol{x} \right)\boldsymbol{h} + \frac{1}{2}\omega  \times \boldsymbol{h} + {\rm O}\left( {h{}^2} \right),\;\;\;h = \left| \boldsymbol{h} \right|
	\label{eq5.3.2}
	\end{equation}
	
	其中 $\mathcal{D} = \frac{1}{2}\left[ {\nabla \boldsymbol{u} + {{\left( {\nabla \boldsymbol{u}} \right)}^t}} \right]$.
	\label{thm5.3.1}
\end{theorem}

\begin{proof}
	
	\kaishu 
	
	由 Taylor 定理
	
	\begin{equation}
	\boldsymbol{u}\left( \boldsymbol{y} \right) = \boldsymbol{u}\left( \boldsymbol{x} \right) + \nabla \boldsymbol{u}\left( \boldsymbol{x} \right)\boldsymbol{h} + {\rm O}\left( {{h^2}} \right)
	\notag 
	\end{equation}
	
	其中 $ \nabla \boldsymbol{u} $ 是 Jacobian 矩阵, 
	
	\begin{equation}
	\nabla \boldsymbol{u} = \left[ {\begin{matrix}
		{{\partial _x}u} & {{\partial _y}u} & {{\partial _z}u}  \\ 
		{{\partial _x}v} & {{\partial _y}v} & {{\partial _z}v}  \\ 
		{{\partial _x}w} & {{\partial _y}w} & {{\partial _z}w}  \\ 
		\end{matrix} } \right]
	\notag 
	\end{equation}
	
	我们将 Jacobian 矩阵 $ \nabla \boldsymbol{u} $ 分解为对称矩阵 $ \mathcal{D} $ 和反对称矩阵 $ \mathcal{S} $ 两部分
	
	\begin{equation}
	\nabla \boldsymbol{u} = \mathcal{D} + \mathcal{S} = \frac{1}{2}\left[ {\nabla \boldsymbol{u} + {{\left( {\nabla \boldsymbol{u}} \right)}^t}} \right] + \frac{1}{2}\left[ {\nabla \boldsymbol{u} - {{\left( {\nabla \boldsymbol{u}} \right)}^t}} \right]
	\label{eq5.3.3}
	\end{equation}
	
	容易验证
	
	\begin{equation}
	\mathcal{S}= \frac{1}{2}\left[ {\nabla \boldsymbol{u}- {{\left( {\nabla \boldsymbol{u}} \right)}^t}} \right],\;\mathcal{S}h = \frac{1}{2}\omega  \times \boldsymbol{h}
	\label{eq5.3.4}
	\end{equation}
	
	代回 Taylor 展开式得 \ref{eq5.3.2}.
	
\end{proof}

\begin{remark}
	矩阵 $ \mathcal{D} $ 的迹(trace) 正好是 $ \boldsymbol{u} $  之散度 $tr\left( \mathcal{D} \right) = div\;\boldsymbol{u}$.
\end{remark}

\begin{definition}[\kaishu 环流]
	$ \boldsymbol{u} $ 是速度场, $ C $ 是一对闭曲线, 则环流(circulation) 定义为
	
	\begin{equation}
	\Gamma \left( t \right) = \oint_C {\boldsymbol{u} \cdot d\boldsymbol{r}} 
	\label{eq5.3.5}
	\end{equation}
	\label{def5.3.2}
\end{definition}

环流基本上是速度场沿曲线的线积分. 直接由定义可得环流 $ \Gamma $之之量纲 $\left[ {\Gamma \left( t \right)} \right] = \left[ \boldsymbol{u} \right]\left[ {d\boldsymbol{r}} \right] = \frac{L}{T}L = \frac{{{L^2}}}{T}$. 

\begin{theorem}[\kaishu Kelvin 环流定理]
	环流是守恒的
	\begin{equation}
	\frac{d}{{dt}}\Gamma \left( t \right) = \frac{d}{{dt}}\oint_C {\boldsymbol{u} \cdot d\boldsymbol{r}}  = 0
	\label{eq5.3.6}
	\end{equation}
\end{theorem}

\begin{proof}
	
	\kaishu
	
	\begin{equation}
	\Gamma \left( t \right) = \oint_C {\boldsymbol{u} \cdot \frac{{d\boldsymbol{r}}}{{ds}}ds}  = \oint_C {\boldsymbol{u} \cdot \boldsymbol{t}ds}
	\notag 
	\end{equation}
	
	$ \boldsymbol{t} $ 是曲线 $ C $ 之单位切向量, $ ds  $ 则是弧长元素
	
	\begin{equation}
	\frac{d}{{dt}}\Gamma \left( t \right) = \oint_C {\frac{d}{{dt}}\left( {\boldsymbol{u} \cdot \boldsymbol{t}} \right)ds}  = \oint_C {\left( {\frac{{d\boldsymbol{u}}}{{dt}} \cdot \boldsymbol{t} + \boldsymbol{u} \cdot \frac{{d\boldsymbol{t}}}{{dt}}} \right)ds} 
	\notag 
	\end{equation}
	
	但是
	\begin{equation}
	\frac{{d\boldsymbol{t}}}{{dt}} = \frac{d}{{dt}}\left( {\frac{{d\boldsymbol{r}}}{{ds}}} \right) = \frac{{d\boldsymbol{u}}}{{ds}}
	\notag 
	\end{equation}
	
	所以
	
	\begin{equation}
	\frac{d}{{dt}}\Gamma \left( t \right) = \oint_C {\frac{d}{{dt}}\left( {\boldsymbol{u} \cdot \boldsymbol{t}} \right)ds}  = \oint_C {\left( {\frac{{d\boldsymbol{u}}}{{dt}} \cdot \boldsymbol{t} + \boldsymbol{u} \cdot \frac{{d\boldsymbol{u}}}{{ds}}} \right)ds} 
	\notag
	\end{equation}
	
	但由 Euler 方程
	
	\begin{equation}
	\color{red}
	\frac{{d\boldsymbol{u}}}{{dt}} =  - \frac{1}{\rho }\nabla P
	\notag 
	\end{equation}
	
	令 $W = \int {\frac{{dP}}{\rho }} $, 则
	
	\begin{equation}
	\nabla W = \nabla \int {\frac{{dP}}{\rho }}  = \frac{d}{{dP}}\left( {\int {\frac{{dP}}{\rho }} } \right)\nabla P = \frac{1}{\rho }\nabla P
	\notag 
	\end{equation}
	
	故
	
	\begin{equation}
	\frac{{d\boldsymbol{u}}}{{dt}} =  - \frac{1}{\rho }\nabla P =  - \nabla W
	\notag 
	\end{equation}
	
	\begin{equation}
	\frac{d}{{dt}}\Gamma \left( t \right) = \oint_C {\left( { - \nabla W \cdot \boldsymbol{t} + \boldsymbol{u} \cdot \frac{{d\boldsymbol{u}}}{{ds}}} \right)ds} 
	\notag 
	\end{equation}
	
	但 $\nabla W \cdot \boldsymbol{t} = \frac{{\partial W}}{{\partial s}}$, 所以
	
	\begin{equation}
	\color{red}
	\frac{{d\Gamma \left( t \right)}}{{dt}} =  - \oint_C {\left[ {\frac{\partial }{{\partial s}}\left( {W - \frac{1}{2}\boldsymbol{u} \cdot \boldsymbol{u}} \right)} \right]ds}  = 0
	\notag 
	\end{equation}
	
	\begin{equation}
	\frac{{d\Gamma \left( t \right)}}{{dt}} = 0 \Rightarrow \Gamma \left( t \right) = \text{\kaishu 常数}
	\notag 
	\end{equation}
	
\end{proof}

\begin{theorem}[\kaishu 涡度方程]
	涡度 $\omega  = \nabla  \times \boldsymbol{u}$ 满足微分方程
	
	\begin{equation}
	\frac{{D\omega }}{{Dt}} + \boldsymbol{\omega} \;div\;\boldsymbol{u}\; = \;\frac{{\partial \boldsymbol{\omega} }}{{\partial t}}\; + \boldsymbol{u} \cdot \nabla \boldsymbol{\omega}  + \boldsymbol{\omega} \;div\;\boldsymbol{u}\; = \;\boldsymbol{\omega}  \cdot \;\nabla \boldsymbol{u}
	\label{eq5.3.7}
	\end{equation}
	
   压力 $ P $ 不出现在涡度方程, 如果是不可压缩流体 $ div \boldsymbol{u} = 0 $ 则涡度方程成为 
   
   \begin{equation}
   \frac{{D \boldsymbol{\omega} }}{{Dt}} = \boldsymbol{\omega}  \cdot \nabla \boldsymbol{u}
   \label{eq5.3.8}
   \end{equation}
   
   不可压缩黏性流体则为
   
   \begin{equation}
   \frac{{D\boldsymbol{\omega} }}{{Dt}} = \boldsymbol{\omega}  \cdot \nabla \boldsymbol{u} + \nu \Delta \boldsymbol{\omega}
   \label{eq5.3.9}
   \end{equation}
	
\end{theorem}

\begin{proof}
	
	\kaishu 
	
	我们从 Euler 方程着手
	
	\begin{equation}
	\frac{{\partial \boldsymbol{u}}}{{\partial t}} + \boldsymbol{u} \cdot \nabla \boldsymbol{u} + \nabla P = 0
	\notag 
	\end{equation}
	
	可以将 $ \boldsymbol{u} \cdot  \nabla \boldsymbol{u} $ 表示为
	
	\begin{equation}
	\color{red}
	\frac{1}{2}\nabla \left( {\boldsymbol{u} \cdot \boldsymbol{u}} \right) = \boldsymbol{u} \cdot \nabla \boldsymbol{u} + \boldsymbol{u} \times \left( {\nabla  \times \boldsymbol{u}} \right) = \boldsymbol{u} \cdot \nabla \boldsymbol{u} + \boldsymbol{u} \times \boldsymbol{\omega} 
	\notag 
	\end{equation}
	
	所以 Euler 方程可以改写为
	
	\begin{equation}
	\frac{{\partial \boldsymbol{u}}}{{\partial t}} - \boldsymbol{u} \times \boldsymbol{\omega}  =  - \nabla \left( {P + \frac{1}{2}{{\left| \boldsymbol{u} \right|}^2}} \right)
	\notag 
	\end{equation}
	
	再取旋度得
	
	\begin{equation}
	\color{red}
	\nabla  \times \left( {\frac{{\partial \boldsymbol{u}}}{{\partial t}} - \boldsymbol{u} \times \boldsymbol{\omega}a } \right) = 0
	\notag 
	\end{equation}
	
	因为 $ \boldsymbol{r} $ 与 $ t $ 是独立变数
	
	\begin{equation}
	\frac{{\partial \boldsymbol{\omega} }}{{\partial t}} = \frac{\partial }{{\partial t}}\left( {\nabla  \times \boldsymbol{u}} \right) = \nabla  \times \frac{{\partial \boldsymbol{u}}}{{\partial t}} = \nabla  \times \left( {\boldsymbol{u} \times \boldsymbol{\omega} } \right) = \boldsymbol{\omega}  \cdot \nabla \boldsymbol{u} - \boldsymbol{u} \cdot \nabla \boldsymbol{\omega}  - \boldsymbol{\omega} \;div\;\boldsymbol{u}
	\notag 
	\end{equation}
	
	但由定义
	
	\begin{equation}
	\frac{{d\boldsymbol{\omega} }}{{dt}} = \frac{{\partial \boldsymbol{\omega} }}{{\partial t}} + \boldsymbol{u} \cdot \nabla \boldsymbol{\omega} 
	\notag 
	\end{equation}
	
	我们可以结论
	
	\begin{equation}
	\color{red}
	\frac{d}{{dt}}\boldsymbol{\omega}  + \boldsymbol{\omega} div\;\boldsymbol{u}\; = \;\omega  \cdot \nabla \boldsymbol{u}
	\notag 
	\end{equation}
\end{proof}

\begin{remark}
	对于二维流体 $\boldsymbol{u} = \left( {\boldsymbol{u}\left( {x,y,t} \right),v\left( {x,y,t} \right)} \right),\boldsymbol{\omega} \left( {0,0,\omega \left( {x,y,t} \right)} \right),\boldsymbol{\omega}  \cdot \nabla \boldsymbol{u} = \boldsymbol{\omega} \frac{{\partial \boldsymbol{u}}}{{\partial z}} = 0$, 涡度方程是一个纯量函数方程
	\begin{equation}
	\frac{{D\omega }}{{Dt}} = 0,\;\frac{{D\omega }}{{Dt}} = \nu \Delta \omega 
	\label{eq5.3.10}
	\end{equation}
	
	对二维流体涡度之变化全依赖于扩散(黏性)作用.
\end{remark}

\begin{theorem}[\kaishu Helmholtz 方程]
	令 $ \boldsymbol{\xi}  = \frac{1}{2}\boldsymbol{\omega}  = \frac{1}{2}\nabla  \times \boldsymbol{u}$, 则
	
	\begin{equation}
	\frac{d}{{dt}}\left( {\frac{\boldsymbol{\xi} }{\rho }} \right) = \frac{\boldsymbol{\xi} }{\rho } \cdot \nabla \boldsymbol{u}
	\label{eq5.3.11}
	\end{equation}
\end{theorem}

\begin{proof}
	
	\kaishu
	
	由连续方程 \ref{eq5.1.7} 得
	
	\begin{equation}
	div \boldsymbol{u} =  - \frac{1}{\rho }\frac{{d\rho }}{{dt}} = \rho \frac{d}{{dt}}\left( {\frac{1}{\rho }} \right)
	\notag 
	\end{equation}
	
	再由涡度方程 \ref{eq5.3.7} 得
	
	\begin{equation}
	\frac{1}{\rho }\frac{{d\boldsymbol{\xi} }}{{dt}} + \boldsymbol{\xi} \frac{d}{{dt}}\left( {\frac{1}{\rho }} \right) = \frac{\boldsymbol{\xi} }{\rho } \cdot \nabla \boldsymbol{u} \Rightarrow \frac{d}{{dt}}\left( {\frac{\boldsymbol{\xi} }{\rho }} \right) = \frac{\boldsymbol{\xi} }{\rho } \cdot \nabla \boldsymbol{u}
	\notag 
	\end{equation}
	
	
\end{proof}


\begin{definition}[\kaishu 速度位]
	流体速度场 $ \boldsymbol{u} $ 之涡度等于零, $\boldsymbol{\omega} = \nabla  \times \boldsymbol{u} = \boldsymbol{u}$ 则称流体是无旋的(irrotational). 黏性(inviscid) 无旋的流体称为位势流(potential flow). 如果在一单一连通区域则存在纯量函数 $ \phi $, 满足
	
	\begin{equation}
	\boldsymbol{u} =  - \nabla \phi ,\;\phi  =  - \int_{{\boldsymbol{r}_0}}^{\boldsymbol{r}} {\boldsymbol{u} \cdot d\boldsymbol{r}} 
	\label{eq5.3.12}
	\end{equation}
	
	我们称 $ \phi $ 是速度位.
	\label{def5.3.3}
\end{definition}

如果是无旋流的流体 $\boldsymbol{u} =  - \nabla \phi $, 则连续方程成为

\begin{equation}
\frac{{d\rho }}{{dt}} - \rho \Delta \phi  = 0
\label{eq5.3.13}
\end{equation}

如果是不可压缩流体, 则 $div\;\boldsymbol{u}\; = \;\Delta \phi  = 0$.

\begin{theorem}
	无旋不可压缩流体的速度位 $ \phi  $ 满足 Laplace 方程 $\Delta \phi  = 0$, 所以 $ \phi  $ 是一调和函数(harmonic function).
	\label{thm5.3.5}
\end{theorem}

\begin{theorem}[\kaishu Bernoulli 方程]
	$\varphi $ 是无旋不可压缩理想气体之速度位, 则 $\varphi $ 满足 Bernoulli 方程:
	\begin{equation}
	\frac{{\partial \varphi }}{{\partial t}} + \frac{1}{2}\nabla \varphi  \cdot \nabla \varphi  + \int {\frac{{dP}}{\rho }}  = F\left( t \right)
	\label{eq5.3.14}
	\end{equation}
	Bernoulli 方程正是动量守恒方程式的积分.
	\label{thm5.3.6}
\end{theorem}

\begin{proof}
	
	\kaishu
	
	利用 $\boldsymbol{u} = \nabla \varphi, div \boldsymbol{u} = 0 $ 代入动量守恒方程式 \ref{eq5.1.15} 得
	
	\begin{equation}
	\nabla \left( {\frac{{\partial \varphi }}{{\partial t}} + \frac{1}{2}\nabla \varphi  \cdot \nabla \varphi } \right) =  - \nabla \int {\frac{{dP}}{{d\rho }}} 
	\notag 
	\end{equation}
	
	因此
	
	\begin{equation}
	\frac{{\partial \varphi }}{{\partial t}} + \frac{1}{2}\nabla \varphi  \cdot \nabla \varphi  + \int {\frac{{dP}}{{d\rho }}}  = 0
	\notag
	\end{equation}
	
\end{proof}

如果是 $ \varphi $ 与时间 $ t $ 无关 则 \ref{eq5.3.11} 可以简化为底下比较常见之形式

\begin{equation}
\frac{1}{2}{\left| {\nabla \varphi } \right|^2} + \frac{P}{\rho } = C (\text{\kaishu 常数})
\label{eq5.3.15}
\end{equation}

如果是均匀流体 $ \rho = $ 常数, 则 Bernoulli 方程式为

\begin{equation}
\frac{1}{2}{\left| \boldsymbol{u} \right|^2} + P = \frac{1}{2}{\left| {\nabla \varphi } \right|^2} + P = \text{\kaishu 常数}
\label{eq5.3.16}
\end{equation} 

\begin{figure}[!htb]
	\centering
	\includegraphics[width=0.5\linewidth,height=0.2\linewidth]{fig_5_3_1.png}
	\label{img002}
	\caption{}
\end{figure}

动能 $ + $ 压力(压强) $ = $ 定数, 这式子说明了为何飞机会上升的原理.

%%%%%%%%%%%%%%%%%%%%%%%%%%%%%%%%%%%%%%%%%%%%%%%%%%%%%%%%%%%%%%%%%% sec 5.4 %%%%%%%%%%%%%%%%%%%%%%%%%%%%%%%%%%%%%%%%%%%%%%%%%%%%%%%%%%%%%%%%%%%%%%%%%%%%%%%%%%%%%%%%%%%%%%%

\newpage 

\subsection{\kaishu 基本电磁学理论}

\subsubsection{\kaishu 库仑定律}

将电荷放在空间中时, 在其存在的位置如果有电力作用于这个电荷的空间叫做电力场, 简称电场. 放在电场内的某一点, 电量 $ q $ 的点电荷所受的电力 $ \boldsymbol{F} $ 与电量成正比, 所以

\begin{equation}
\boldsymbol{E} = \frac{\boldsymbol{F}}{q}
\label{eq5.4.1}
\end{equation}

这是那个点所定的物理量, 叫做电场强度, 意思是单位点电荷所受的作用力. 

在自然界有两种电荷, 异性相吸,同性相斥. 而电荷间作用的力和电量之乘积成正比, 与距离的平方成反比, 此定律 1776 年由英国物理学家 H. Cavendish (1731--1810) 发现, 1785 年由库伦 (Charles Augustin de Coulomb, 1736--1806) 从实验所确定. 以数学形式表示则为 

\begin{equation}
\boldsymbol{F} \propto \frac{{{q_1}{q_2}}}{{{r^2}}},\;\boldsymbol{F} = k\frac{{{q_1}{q_2}}}{{{r^2}}}
\label{eq5.4.2}
\end{equation}

此式称库仑定律. $ k $ 为一常数, 其值与所用的单位有关. 由 \ref{eq5.4.2} 可得电场

\begin{equation}
\boldsymbol{E}\left( \boldsymbol{x} \right) = \frac{q}{{{{\left| \boldsymbol{x} \right|}^2}}}\frac{\boldsymbol{x}}{{\left| \boldsymbol{x}x \right|}} = q\frac{\boldsymbol{x}}{{{{\left| \boldsymbol{x} \right|}^3}}}
\label{eq5.4.3}
\end{equation}

如果有不同的点电荷, 则利用重叠原理(线性)将各点都加起来

\begin{equation}
\boldsymbol{E}\left( \boldsymbol{x} \right) = \sum\limits_i {{q_i}\frac{{\boldsymbol{x} - {\boldsymbol{x}_i}}}{{{{\left| {\boldsymbol{x} - {\boldsymbol{x}_i}} \right|}^3}}}} 
\label{eq5.4.4}
\end{equation}

假设电荷分布(charge distribution) 是连续的, 则 \ref{eq5.4.4} 可写为积分形式

\begin{equation}
\boldsymbol{E}\left( \boldsymbol{x} \right) = \iiint {\rho \left( \boldsymbol{y} \right)\frac{{\boldsymbol{x} - \boldsymbol{y}}}{{{{\left| {\boldsymbol{x} - \boldsymbol{y}} \right|}^3}}}d\boldsymbol{y}}
\label{eq5.4.5}
\end{equation}

这可以视为广义库伦定律或积分形式的库伦定律, 若 $ \rho $ 已知, 则由 \ref{eq5.4.5} 可得所有静电场的问题. 但问题是大部分情形只有一部分的电荷分布为已知, 因此借助于微分方程乃是必须的. 由  Helmholtz 定理 \ref{thm5.3.6} 我们仅需知道电场 $ \boldsymbol{E} $ 的散度(divergence) 与旋度(curl). 利用关系式 $\nabla \frac{1}{{\left| {\boldsymbol{x} - \boldsymbol{y}} \right|}} =  - \frac{{\boldsymbol{x} - \boldsymbol{y}}}{{{{\left| {\boldsymbol{x} - \boldsymbol{y}} \right|}^3}}}$, 库伦定律 \ref{eq5.4.5} 可改写为
\begin{equation}
\boldsymbol{E}\left( \boldsymbol{x} \right) = \iiint {\rho \left( \boldsymbol{y} \right)\frac{{\boldsymbol{x} - \boldsymbol{y}}}{{{{\left| {\boldsymbol{x} - \boldsymbol{y}} \right|}^3}}}dy} =  - \nabla \iiint {\frac{{\rho \left( \boldsymbol{y} \right)}}{{\left| {\boldsymbol{x} - \boldsymbol{y}} \right|}}d\boldsymbol{y}}
\label{eq5.4.6}
\end{equation}

这告诉我们电场 $ \boldsymbol{E} $ 可表为某个数值函数(scalar function) 的梯度(gradient), 由此自然而然可定义电位能 $ V(\boldsymbol{x}) $

\begin{equation}
V\left( \boldsymbol{x} \right) \equiv \iiint {\frac{{\rho \left( \boldsymbol{y} \right)}}{{\left| {\boldsymbol{x} - \boldsymbol{y}} \right|}}d\boldsymbol{y}}
\label{eq5.4.7}
\end{equation}

所以库伦定律就成为

\begin{equation}
\boldsymbol{E}\left( \boldsymbol{x} \right) =  - \nabla V\left( \boldsymbol{x} \right)
\label{eq5.4.8}
\end{equation}

利用 \ref{eq5.4.8} 可以容易计算散度(divergence) 与旋度(curl). 首先是旋度

\begin{equation}
curl \boldsymbol{E} = \nabla \times \boldsymbol{E} = - \nabla \times \nabla V =0
\label{eq5.4.9}
\end{equation}

至于散度则需要散度定理(Gauss 定理), 先看单个点电荷

\begin{equation}
\iint {\boldsymbol{E} \cdot d\boldsymbol{S}} = \iint {E\;\cos \;\theta \;dS} = \int {\frac{{q\cos \;\theta }}{{{r^2}}}dS}  = 4\pi q
\label{eq5.4.10}
\end{equation}

若是电荷分布是连续则 \ref{eq5.4.10} 成为

\begin{equation}
\iint {\boldsymbol{E} \cdot d\boldsymbol{S}} = 4\pi \iiint {\rho \left( \boldsymbol{x} \right)d\boldsymbol{x}}
\label{eq5.4.11}
\end{equation}

由 Gauss 定理

\begin{equation}
\iint {\boldsymbol{E} \cdot d\boldsymbol{S}} = \iiint {div\;\boldsymbol{E}d\boldsymbol{x}} = \iiint {\nabla  \cdot \boldsymbol{E}d\boldsymbol{x}}
\label{eq5.4.12}
\end{equation}

所以

\begin{equation}
\iiint {div\;\boldsymbol{E}d\boldsymbol{x}} = 4\pi \iiint {\rho \left( \boldsymbol{x} \right)d\boldsymbol{x}}
\label{eq5.4.13}
\end{equation}

因此可得微分形式的库伦定律

\begin{equation}
div\;\boldsymbol{E}\; = \;4\pi \rho \quad (\text{\kaishu 库仑定律})
\label{eq5.4.14}
\end{equation}

库伦定律就是电学的高斯定律, 我们可以理解为有电荷 $ \rho $ 才会产生电场 $ E $. 为了方便将 \ref{eq5.4.9}, \ref{eq5.4.14} 合并

\begin{equation}
\left\{ {\begin{matrix}
	{div\;\boldsymbol{E}\; = 4\pi \rho }  \\ 
	{\nabla  \times \boldsymbol{E} = \boldsymbol{0}}  \\ 
	\end{matrix}} \right.
	\notag 
\end{equation}

借由电位能 $ V(\boldsymbol{E} = - \nabla V) $ 则上式成为 Poisson 方程

\begin{equation}
\Delta V =  - 4\pi \rho 
\label{eq5.4.15}
\end{equation}


\subsubsection{\kaishu 安培定律}

稳定电流通过线圈而产生的磁场是可测量的. 静磁场在任意围道(contour)的线积分(line integral)与电流成正比而与围道无关, 这就是安培定律(Ampere's law):

\begin{equation}
\oint {\boldsymbol{B} \cdot d\boldsymbol{r}}  = \frac{{4\pi }}{c}\boldsymbol{I}
\label{eq5.4.16}
\end{equation}

其中比例常数 $\frac{{4\pi }}{c}$($ c $: 光速) 是由实验而得. 我们仍然仿电场计算磁场 $ \boldsymbol{B} $ 的散度与旋度. 由于不可能创造孤立磁极(magnetic monopoles), 因此

\begin{equation}
div \boldsymbol{B} = \nabla \cdot \boldsymbol{B} = 0
\label{eq5.4.17}
\end{equation}

散度是测量场源的局部密度变化, 而由磁场的特性可知无论区域大小如何, 磁极总是有相同的北极(N)与南极(S), 故 $ div \boldsymbol{B} = \nabla \cdot \boldsymbol{B} = 0 $.

至于旋度则可透过安培定律而来

\begin{equation}
\oint {\boldsymbol{B} \cdot d\boldsymbol{r}}  = \frac{{4\pi }}{c}I = \frac{{4\pi }}{c}\iint {\boldsymbol{J} \cdot d\boldsymbol{S}}
\label{eq5.4.18}
\end{equation}

其中 $ \boldsymbol{J} $ 是电流密度(current density), 由 Stokes 定理

\begin{equation}
\oint {\boldsymbol{B} \cdot d\boldsymbol{r}}  = \iint {\nabla  \times \boldsymbol{B} \cdot d\boldsymbol{S}}
\label{eq5.4.19}
\end{equation}

所以 \ref{eq5.4.18} 化为

\begin{equation}
\iint {\nabla  \times \boldsymbol{B} \cdot d\boldsymbol{S}} = \frac{{4\pi }}{c}\iint {\boldsymbol{J} \cdot d\boldsymbol{S}}
\notag 
\end{equation}

因此微分形式的安培定律

\begin{equation}
curl\;\boldsymbol{B} = \nabla  \times \boldsymbol{B} = \frac{{4\pi }}{c}\boldsymbol{J}, \quad  (\text{\kaishu 安培定律})
\label{eq5.4.20}
\end{equation}

\ref{eq5.4.17}, \ref{eq5.4.20} 就是静磁场所遵循的微分方程

\begin{equation}
\nabla  \times \boldsymbol{B} = \frac{{4\pi }}{c}\boldsymbol{J},\;\;\;\nabla  \cdot \boldsymbol{B} = 0
\label{eq5.4.21}
\end{equation}

安培定律是磁学的高斯定律, 所谓静电场或静磁场之值不随时间而变化

\begin{equation}
\frac{{d{\boldsymbol{E}}}}{{dt}} = \mathop {\boldsymbol{E}}\limits^ \cdot   = \frac{{d{\boldsymbol{B}}}}{{dt}} = \mathop {\boldsymbol{B}}\limits^ \cdot   = \boldsymbol{0}.
\notag 
\end{equation}

\subsubsection{\kaishu 法拉第定律}

电场与磁场的关系是透过法拉第定律(电动势等于穿过导电电路的磁通量的变化率)来联系

\begin{equation}
\oint_C {\boldsymbol{E} \cdot d\boldsymbol{r}}  =  - \frac{\partial }{{\partial t}}\frac{1}{c}\iint_S {\boldsymbol{B} \cdot d\boldsymbol{S}},\;\;\;C = \partial S
\label{eq5.4.22}
\end{equation}

由 Stokes 定理

\begin{equation}
\oint_C {\boldsymbol{E} \cdot d\boldsymbol{r}}  = \iint_S {\nabla  \times \boldsymbol{E} \cdot d\boldsymbol{S}} =  - {\partial _t}\iint_S {\frac{1}{c}\boldsymbol{B} \cdot d\boldsymbol{S}}
\label{eq5.4.23}
\end{equation}

因此

\begin{equation}
\nabla  \times \boldsymbol{E} =  - \frac{1}{c}{\partial _t}\boldsymbol{B}
\label{eq5.4.24}
\end{equation}

此时描写电磁,磁场的方程如下

\begin{equation}
\left\{ {\begin{matrix}
	{\nabla  \times \boldsymbol{E} =  - \frac{1}{c}{\partial _t}\boldsymbol{B},}  \\ 
	{\nabla  \cdot \boldsymbol{E} = 4\pi \rho ,}  \\ 
	\end{matrix} } \right. \quad \begin{matrix}
{\nabla  \times \boldsymbol{B} = \frac{{4\pi }}{c}\boldsymbol{J}}  \\ 
{\nabla  \cdot \boldsymbol{B} = 0}  \\ 
\end{matrix} 
\label{eq5.4.25}
\end{equation}

这是静电场与静磁场之推广. 

\subsubsection{\kaishu 位移电流与 Maxwell 方程}

物理学家的基本定律说到电荷是不可毁灭的, 它永远不会消失也不会被创造出来. 1860 年前后, Maxwell 发现 \ref{eq5.4.25} 这组方程式还缺少某些重要的东西. ({\color{blue} 实际上这组方程式最大的缺陷是违反电荷守恒律(conservation law of charge)}). 为此我们需要由电荷守恒律导出连续性方程(equation of continuity). 由散度定理

\begin{equation}
\iiint_\Omega  {div\;\boldsymbol{J}d\boldsymbol{x}\; = \;\iint_{\partial \Omega } {\boldsymbol{J} \cdot d\boldsymbol{S}}}
\notag 
\end{equation}

右式这个曲面积分的物理意义代表全部电流(current)流入或流出曲面 $ {\partial \Omega } $ 的量即通量(flux), 但由电荷守恒律知通量等于整个体积 $ \Omega $ 内质量的改变量

\begin{equation}
\iiint_\Omega  {div \ \boldsymbol{J}d\boldsymbol{x} = \iint_{\partial \Omega } {\boldsymbol{J} \cdot d\boldsymbol{S}}} =  - \frac{\partial }{{\partial t}}\iiint_\Omega  {\rho d\boldsymbol{x}} =  - \iiint_\Omega  {\frac{\partial }{{\partial t}}\rho d\boldsymbol{x}}
\label{eq5.4.26}
\end{equation}

因为对任意的 $ \Omega $ 上式皆成立所以 $ \rho, \boldsymbol{J} $ 满足连续性方程

\begin{equation}
\frac{{\partial \rho }}{{\partial t}} + div \ \boldsymbol{J} = \boldsymbol{0}
\label{eq5.4.27}
\end{equation}

另一方面, 由安培定律 \ref{eq5.4.20} 两边取旋度得

\begin{equation}
div\;\boldsymbol{J}\; = \frac{c}{{4\pi }}div\left( {\nabla  \times \boldsymbol{B}} \right) = \boldsymbol{0}
\label{eq5.4.28}
\end{equation}

比较连续方程

\begin{equation}
\frac{c}{{4\pi }}div\left( {\nabla  \times \boldsymbol{B}} \right) = 0 = \frac{\partial }{{\partial t}}\rho  + div\;\boldsymbol{J}
\notag 
\end{equation}

但 $\rho  = \frac{1}{{4\pi }}\nabla  \cdot \boldsymbol{E}$ (假设静电场方程式成立), 则

\begin{equation}
\frac{c}{{4\pi }}\nabla  \cdot \left( {\nabla  \times \boldsymbol{B}} \right) = \nabla  \cdot \boldsymbol{J} + \frac{1}{{4\pi }}\nabla  \cdot {\partial _t}\boldsymbol{E}
\label{eq5.4.29}
\end{equation}

所以 $ \boldsymbol{B} $ 之旋度 $ curl \boldsymbol{B} = \nabla \times \boldsymbol{B} $ 满足的方程式为

\begin{equation}
\nabla  \times \boldsymbol{B} = \frac{{4\pi }}{c}\boldsymbol{J} + \frac{1}{c}{\partial _t}\boldsymbol{E}
\label{eq5.4.30}
\end{equation}

将 \ref{eq5.4.25} 式重新改写为

\begin{equation}
\left\{ {\begin{matrix}
	{\nabla  \times E =  - \frac{1}{c}{\partial _t}B}  \\ 
	{\nabla  \times B = \frac{{4\pi }}{c}J + \frac{1}{c}{\partial _t}E}  \\ 
	{\nabla  \cdot E = 4\pi \rho }  \\ 
	{\nabla  \cdot B = 0}  \\ 
	\end{matrix} } \right. \quad \begin{matrix}
{\text{\kaishu 法拉第定律}}  \\ 
{\text{\kaishu 安培定律, 电荷守恒}}  \\ 
{\text{\kaishu 库仑定律}}  \\ 
{} \\ 
\end{matrix} 
\label{eq5.4.31}
\end{equation}

借着这个附加项, $\frac{{4\pi }}{c}\boldsymbol{J}$ 即所谓的位移电流来补充, 就可以克服对 \ref{eq5.4.25} 这些方程式进行数学变换, 得出一种与 Maxwell 方程系得其他公式矛盾得关系式. 因此, Maxwell 就从法拉第得解释者成为有独创性得学者. 这四个方程式就称为 Maxwell 方程(在真空), 是电磁学最基本的微分方程.

由 Maxwell 方程 \ref{eq5.4.31} 直接可得的推论就是电场 $ \boldsymbol{E} $ 与磁场 $ \boldsymbol{B} $ 都满足波动方程(wave equation), 对任意的向量 $ \boldsymbol{F} $, 下式成立

\begin{equation}
\Delta \boldsymbol{F} =  - \nabla  \times \left( {\nabla  \times \boldsymbol{F}} \right) + \nabla \left( {\nabla  \cdot \boldsymbol{F}} \right)
\label{eq5.4.32}
\end{equation}

以三度空间来说

\begin{equation}
\Delta \boldsymbol{E} = {\left( {\Delta {F_1},\Delta {F_2},\Delta {F_3}} \right)^t}
\notag 
\end{equation}

向量场 $ \boldsymbol{F} $ 分布以电场 $ \boldsymbol{E} $ 与磁场 $ \boldsymbol{B} $ 代入并利用 Maxwell 方程

\begin{equation}
\Delta \boldsymbol{E} =  - \nabla  \times \left( { - \frac{1}{c}\frac{{\partial \boldsymbol{B}}}{{\partial t}}} \right) + 4\pi \nabla \rho  = \frac{1}{{{c^2}}}\frac{{{\partial ^2}\boldsymbol{E}}}{{\partial {t^2}}} + \frac{{4\pi }}{{{c^2}}}\frac{{\partial \boldsymbol{J}}}{{\partial t}} + 4\pi \nabla \rho 
\label{eq5.4.33}
\end{equation}

\begin{equation}
\Delta \boldsymbol{B} =  - \nabla  \times \left( { - \frac{1}{c}\frac{{\partial \boldsymbol{E}}}{{\partial t}} + \frac{{4\pi }}{c}\boldsymbol{J}} \right) = \frac{1}{{{c^2}}}\frac{{{\partial ^2}\boldsymbol{B}}}{{\partial {t^2}}} - \frac{{4\pi }}{{{c^2}}}\nabla  \times \boldsymbol{J}
\label{eq5.4.34}
\end{equation}

换句话说, $ \boldsymbol{E},\boldsymbol{B} $ 满足非齐次波动方程

\begin{equation}
{\boldsymbol{E}_{tt}} - {c^2}\Delta \boldsymbol{E} =  - 4\pi \left( {{c^2}\nabla \rho  + {\boldsymbol{J}_t}} \right)
\label{eq5.4.35}
\end{equation}

\begin{equation}
{\boldsymbol{B}_{tt}} - {c^2}\Delta \boldsymbol{B} = 4\pi c\left( {\nabla  \times \boldsymbol{J}} \right)
\label{eq5.4.36}
\end{equation}

如果考虑的区域没有电荷或电流, 则 $ \boldsymbol{E},\boldsymbol{B} $ 满足齐次波动方程.

\begin{equation}
{\boldsymbol{E}_{tt}} - {c^2}\Delta \boldsymbol{E} = {\boldsymbol{B}_{tt}} - {c^2}\Delta \boldsymbol{B} = 0
\label{eq5.4.37}
\end{equation}

\ref{eq5.4.35},\ref{eq5.4.36} 还告诉我们电场,磁场是以光速 $ c $ 传播, 也因此我们习惯通称为电磁波. 
%%%%%%%%%%%%%%%%%%%%%%%%%%%%%%%%%%%%%%%%%%%%%%%%%%%%%%%%%%%%%%%%%%%%% sec 5.5 %%%%%%%%%%%%%%%%%%%%%%%%%%%%%%%%%%%%%%%%%%%%%%%%%%%%%%%%%%%%%%%%%%%%%%%%%%%%%%%%%%%%%%%%%%%

%\newpage 

%\subsection{\kaishu 电磁学之量纲(因次)}

%%%%%%%%%%%%%%%%%%%%%%%%%%%%%%%%%%%%%%%%%%%%%%%%%%%%%%%%%%%%%%%%%%%%% 参考文献 %%%%%%%%%%%%%%%%%%%%%%%%%%%%%%%%%%%%%%%%%%%%%%%%%%%%%%%%%%%%%%%%%%%%%%%%%%%%%%%%%%%%%%%%%%%

\begin{center}
	\large 参考文献
\end{center}

\vspace{-1cm}

\beginrefs
\bibentry{1}{\kaishu  林琦焜}, 
``向量分析,''
{\it 沧海书局 },
民国 96 年.
%\bibentry{2}{ P. WU} , 
%`` Real Analysis,''
%{\url{https://ir.nctu.edu.tw/handle/11536/108280}},
%2010.
\endrefs

%%%%%%%%%%%%%%%%%%%%%%%%%%%%%%%%%%%%%%%%%%%%%%%%%%%%%%%%%%%%%%%%%%%%%%%%%%%%%%%%%%%%%%%%%%%%%%%%%%%%%%%%%%%%%%%%%%%%%%%%%%%%%%%%%%%%%%%%%%%%%%%%%%%%%%%%%%%%%%%%%%%%%%%%%%
\begin{flushright}
	\tiny \today 
\end{flushright}


\end{document}

              